\title{Równianie Kuramoto-Sivashinskiego}

%Będziemy wykorzystywali wcześniej wprowadzone twierdzenia do wykazania $ \Sigma_2$ chaous w układzie równań $ Michelson'a $
\section{Równanie Kuramoto-Sivashinskiego}
Rozważmy równanie :


\begin{equation}\label{eq:kura}
  u_t + \nabla^4u + \nabla^2u + \frac{1}{2}| \nabla u|^2 = 0
 \end{equation}

Michelson[cyt] przeprowadził badanie tego równania. Zauważył, że wcześniejsze badania [cyt] pokazują, ze jeśli równanie \ref{eq:kura} 
ma rozwiązanie na dużym przedziale symetrycznym $ (-l,l) $ z okresowym warunkiem brzegowym to rozwiązania przyjmują postać 
$ u(x,t) = -(c_0)^2t + v(x,t) $ gdzie $ c_0 \approx 1.04 $ oraz $ v $ jest quasi-okresową falą. Dlatego przyjął, że $ v $ nie zależy od 
$ t $ i rozważał rozwiązania równania \ref{eq:kura} postaci $ u(x,t) = - c^2t + v(x) $. Jeżeli podstawimy $ y = v'$ to możemy przekształcić \ref{eq:kura}
do postaci 
\begin{equation}
\label{eq:jednazmienna}
  y^{'''} + y' = c^2 - \frac{1}{2}y^2
\end{equation}

Co można zapisać jako układ równań nazywany układem Michelsona

\begin{equation}\label{eq:michealsonSystem}
\left\{
    \begin{array}{rcl}
        \dot x &=& y\\
        \dot y &=& z\\
        \dot z &=& c^2 - \lambda y -\frac{1}{2}x^2
    \end{array}
\right.
\end{equation}



$$
  u_t = -u_{xxxx} - u_{xx} -uu_x
$$


Układ posiada dwa punkty stacjonarne : $x_{-}(c) = (-c\sqrt{2},0,0) $ i $x_{-}(c) = (c\sqrt{2},0,0)$
Co więcej zachowuje miarę i odwzorowanie : $ {\mathbb R} ^4: \to {\mathbb R}^4 $ dane przez:

\begin{equation}
   R(x,y,z,t) = (-x,y,-z,-t)
\end{equation}


\subsection{Wybrane znane wyniki} 

\begin{theorem}[Troy]
 
Rozważmy równanie \ref{eq:jednazmienna} z parametrem c = 1. Wtedy istnieją co najmniej dwa symetryczne(nieparzyste) rozwiązanai okresowe
spełniające warunek
$$
    y(0) = y''(0) = 0
$$
\end{theorem}

\begin{theorem} [Troy]
 Rozważmy równanie \ref{eq:jednazmienna} z parametrem c = 1. Wtedy istnieją co najmniej dwie symetryczne (nieparzyste) orbity
 heterokliniczne pomiędzy punktami stacjonarnymi $(y,y',y'') = (\pm \sqrt{2},0,0)$
 
\end{theorem}
Dla równania \eqref{eq:michealsonSystem}, chciałbym przytoczyć następujące wyniki przez naukowców z naszego wydziału.


\begin{theorem} [Mrozek, Żelawski]
  Dla parametru c = 1 oraz dla każdego $ \lambda \in [0,1] $ istnieje rozwiązanie heterokliniczne łączące punkt równowagi $(\sqrt{2},0,0) $ 
  oraz $(-\sqrt{2},0,0)$
 
\end{theorem}

Kładąc w rówaniu \eqref{eq:michealsonSystem} $ \lambda = 1 $
Daniel Wilczak uzyskał między innymi następujący wyniki.

\begin{theorem}[Wilczak]
  Dla każdego parametru $ c \in [0.8285,0.861] $ układ Michelsona jest $ \Sigma_4 $ chaotyczny,
  czyli, że istnieje odwzorowanie Poincar\'ego takie, że jest pół sprzężone z pełnym shiftem na czterech symbolach.
\end{theorem}


\subsection{Symetria w równaniach Kuramoto-Sivashinskiego}

W równaniach Kuramoto-Sivashinskiego występuje następująca symetria odwracania czasu 

$$
  R(x,y,z) = (-x,y,-z).
$$
Z istnienia tej symetrii wywnioskujemy, że dowolne rozwiązanie o warunku początkowym $x(0) = z(0) = 0 $ 
jest symetryczne względem osi y. Ponadto jeśli takie rozwiązanie dwukrotnie przetnie oś y to musi być rozwiązaniem okresowym.



W tej pracy będziemy badać równianie \eqref{eq:michealsonSystem} dla parametru $ \lambda = 1$.
Zauważmy, że pole wektorowe równania \eqref{eq:michealsonSystem} jest styczne do płaszczyzny $(x,y,0) $
tylko na paraboli 
$$
      L := \{ (x,c^2 - \frac{1}{2} x^2,0) | x \in \mathbb R\}.
$$

W związku z tym możemy określić lokalną sekcje równania \eqref{eq:michealsonSystem} jako 
$$
  \Theta := \{(x,y,0) | x,y \in \mathbb R \} .
$$

Sekcja $ \Theta $ wybraliśmy w ten sposób by trzecia współrzędna była stale równa 0, 
a więc by określić punkt na niej musimy podać jedynie współrzędne (x,y) i tak od tej pory będziemy czynić.
Zauważmy również, że $ R(\Theta) = \Theta $ a więc z lematu \eqref{lem:poincareReverse} istnieje symetria odwrócenia czasu 
dla odwzorowania Poincar\'ego
$ P : \Theta \oarrow \Theta $. Będziemy ją również oznaczać R. Nie będzie to prowadziło do żadnych niejednoznaczności 
ponieważ od tej pory będziemy
zajmowali się jedynie dynamiką na $ \Theta$ .
Reasumując nasze odwzorowanie $ P : \Theta \oarrow \Theta $ spełnia następujący warunek:

\begin{equation}
  \forall u \in dom(P) \quad R(u) \in dom(P^{-1}) , R(P(u)) = P^{-1}(R(u)),
\end{equation}
dla $ R(x,y) = (-x,y) $ 

Głównym twierdzeniem, które wykaże w tej pracy 
\begin{theorem}
  Dla każdego parametru $ c = 0.49 $ układ Michelsona jest $ \Sigma_4 $ chaotyczny,
  czyli, że istnieje odwzorowanie Poincar\'ego takie, że jest pół sprzężone z pełnym shiftem na czterech symbolach.
\end{theorem}

