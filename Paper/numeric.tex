\title{Numeryka}

\section{Arytmetyka Przedziałowa}
W tym rozdziale chciałbym przedstwiać techniczne podstawy dowodów komputerowo wspieranych
w dynamice. W jak spsób korzystając z arytmetyki liczb wymiernych o skończonym rozwinięciu
jesteśmy wstanie udowdnić twierdzenia dotyczące elementów z przestrzeni $ \mathbb R^n$ 

\subsection{Liczby reprezentowalne}

W obliczeniach przeprowadzonych na komputerzez przeważnie korzysta się z liczb zmiennoprzecinkowych typu double 
zgodnych ze standarder IEEE 754. Standard ten jest zaimplementowany w 
wielu procesorach, w sczególnosci w komputerach typu PC.
Opisuję on między innymi sposby zaokrąglania liczb czy zachowanie w przypadku dzielenia przez zero.
  Każda liczba typu double jest zakodowana jako 64-bitowy ciąg, w którym 54 młodsze bity kodują mantyse (ozn. $m$) a kolejne
11 bitów koduje wykładnik(ozn. $w$), natomiast najstarszy bit jest bitem znaku (ozn. z). Wtedy wartość tak zakodowanej liczby typu
double wyliczamy z następującego wzoru :

\begin{equation}\label{eq:kodowanie}
  x = (-1)^z*m*2^{w+1}
\end{equation}

Tu może jakiś przykład ?, generaalnie na wikipedii piszą, ż


e m ma być libczą naturalną??


Zbiór wszystkich liczb, które można przedstawić w ten sposób będzie oznaczany $ \hat{\mathbb{R}} $ a jego 
liczby będziemy nazywać {\em liczbami reprezentowalnymi}. Przez $ \overline{\mathbb R}$ będziemy ozanczać
zbiór liczb rzeczywistych rozszerzony o $ \pm \inf$ ze standardowym porządkiem.

Jak już wspominałem wcześniej nie wszystkie liczby rzeczywiste możemy zakodować tak jak podano w \ref{eq:kodowanie}.
Liczby rzeczywiste, które nie są liczbami reprezentowalnymi możemy przybliżać następującymi suriekcjiami

$ \downarrow \uparrow : \mathbb R \to \hat{\mathbb{R}} $
$$
  \downarrow (x) := max \{ y \in \hat{\mathbb{R}} | y \leq x \}
$$
$$
  \uparrow (x) := min \{ y \in \hat{\mathbb{R}} | y \leq x \}
$$

Wybór któreś z podanych suriekcji, która za pomocą której chcemy przybliżać liczby naturalne jest możliwy poprzez 
ustawienie odpowiedniej flagi na procesorze.
Istotną rodziną zbiorów w $ \mathbb R $ jest rodzina przedziałów. Będziemy ją oznaczać poprzez $ \mathbb I $. 
Jej podzbiór, rodzinę przedziałów o końcach którymi są liczby reprezentowalne będziemy oznaczać poprzez $ \hat{\mathbb I } $.
Poprzez $ \overline { \mathbb I } $ będziemy oznaczać rodzine przedziałów której końce należą do $  \overline{\mathbb R} $

Dla $ x \in \mathbb R $ definiujemy najmniejszy przedział reprezentowalny zawierający $ x $ przy pomocy funkcji
$ \updownarrow : \mathbb R \to \hat{\mathbb R } $ jako
$$
    \updownarrow (x) := [\downarrow (x), \uparrow (x) ]\in \hat{\mathbb I } 
$$

W oczywisty sposób funkcje $ \uparrow, \downarrow, \updownarrow $ można zdefinować na podzbiorach zbioru liczb rzeczywistych.
Dla podzbioru $ A \subset \mathbb R $ 
\begin{eqnarray*}
    \uparrow (A) &:=& max \{ \uparrow (a) | in a \in A \} \\
    \downarrow (A) &:=& min \{ \downarrow (a) | in a \in A \} \\
    \updownarrow (A) &:=& [\downarrow(A),\uparrow(A)] \\
\end{eqnarray*}

\section{Arytmetyka przedziałowa}
Obliczenia na komputerach są jak już wcześniej wspominałem wykonywane najczęściej na liczbach typu $ double $.
Oczywiście wynik operacji dwóch liczb reprezentowalnych wcale nie musi być liczbą reprezentowalną.
Jednak chcielibyśmy wykonywać obliczenia ścisłe i by wyniki operacji były wiarygodne i o ile to możliwe precyzyjne.
Dlatego rosrszerzamy operacje dodwania, odejmowania i dzielenia ze zbioru liczb rzeczywistych na rodzine przedziałów 
reprezentowalnych.

Niech $ A,B  \subset \overline {\mathbb R } $ będą pewnymi podzbiorami oraz niech $ \diamondsuit $ będzie jedną z operacji elementarnych,
 $ \diamondsuit \in \{+,-,\cdot,/ \}.$ Wtedy definujemy 
 \begin{equation}
    A \diamondsuit B := \{ a \diamondsuit b | a \in A , b \in B  \}.
 \end{equation}

 Dodatkowo zakładamy, że przy dzieleniu $0 \notin B $. W sytuacji w której $ A,B \in \overline{ \mathbb I } $ operacje te sprowadzja 
 sie do odpowiednich obliczeń na końcach przedziałów, oraz sprawdzenia znaków liczb, będącymi końcami przedziałów $ A,B $.
 Wynik takiej operacji jest również przedziełem $ A \diamondsuit B \in \hat {\mathbb R } $ .
 Możemy próbować w ten sposób rozszerzyć podstawowe operacje na przedziały reprezentowalne. Oczywiście wynik takiej operacji
 niekoniecznie musi być przedziałem reprezentowalnym, dlatego opreacje elementerne na przedziałach reprezentowalnych musimy zdefiniować w
 następujący spsób 
 \begin{equation}
    A \hat{\diamondsuit} B := \updownarrow (A \diamondsuit B ) 
 \end{equation}
 gdzie $ \hat { \diamondsuit } $ oznacza operacje elementarną w {\em arytmetyce przedziałowej }. 
 W ten sposób zdefiniowane operacje arytmetyczne zapewniają, że dla $ X,Y \subset \mathbb R $ zachodzi
 \begin{equation}
    X \diamondsuit Y \subset \updownarrow(X) \hat{\diamondsuit} \updownarrow(Y)
 \end{equation}
 
 o ile prawa strone isntieje. 
 Operacje na końcach są standartowymi operacjami wykonywanymi sprzętowo. Dzięki temu cechują się one 
 wysoką wydajnością. 
 
 W mojej pracy jak już wcześniej pisałem będę korzystał z arytmetyki przedziałowej i metod numerycznych zaimplementowanych w Instytucie Informatyki Uniwersytetu 
 Jagiellońskiego przez grupę badawczą CAPD \cite{CAPD}.

 


% tu Daniel ma komentarz o porównywaniu zbiorów , zobacze czy to będzie istotne