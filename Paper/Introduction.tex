\section{Wstęp}

Matematycy już od XVII wieku próbują rozwiązywać różne równania różniczkowe.
Czasami są wstanie wyprowadzić jawne wzory, jak na przykład dla równań liniowych. Niestety w większości
przypadków nie da się ich wyprowadzić.
Często rozważają jedynie istnienie pewnych rodzajów rozwiązań np okresowych, heteroklicznych,
homoklinicznych czy chaotycznych. Niestety dowody korzystające z metod klasycznych przeważnie mają bardzo 
sztywne założenia co do prawej strony równania, na przykład często zakłada się że jest ono okresowe.

Natomiast w wiele klasycznych równaniach różniczkowych, takich jak równanie Lorenza, R\"{o}sslera czy Kuramoto-Sivashinskiego wykazuje 
zachowanie chaotyczne w symulacjach komputerowych, których nie potrafimy wykazać analitycznie.
W 1995 Mischaikow i Mrozek \cite{MM} udowodnili istnienie dynamiki chaotycznej 
w równaniach Lorenza. Praca w której ukazał się ten wynik została przez Encyklopedie Britannica uznana za jedną z czterech najważniejszych 
prac matematycznych w 1995 roku. Zgliczyński [cyt] wykazał istnienie dynamiki chaotycznej w równaniu R\"{o}sslera.
Dla równania Kuramoto-Sivashinskiego istnieją dowody wspierane komputerowo istnienia 
orbit heteroklinicznych,homoklinicznych[cyt]. O tym ostatnim równaniu napiszę więcej wreszcie pracy, dlatego nie przytaczam tutaj żadnych nazwisk.
Wszystkie cytowane przeze dowody były wspierane komputerowo.

W tej pracy przedstawię pojęcie relacji nakrywających. Przytoczę odpowiednie definicje. Na końcu wykaże, że istnieje $ \Sigma_2 $ chaosu w równaniu Kuramato-Sivashinskiego dla 
pewnej wartości parametru. Będę korzystał z metod numerycznych zaimplementowanych w CAPD \cite{CAPD}














