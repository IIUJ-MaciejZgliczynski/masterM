\documentclass[10pt,a4paper]{article}

\usepackage[utf8]{inputenc}
\usepackage[T1]{fontenc}
\usepackage{amssymb}
\usepackage[polish]{babel}
\usepackage{amsmath}
\usepackage{graphicx}
\usepackage{epstopdf}
\usepackage{epsfig}
\def\oarrow{{\mathrel{\longrightarrow\mkern-25mu\circ}\;\;}}
\newcommand{\Id}{\mathrm{Id}}
 \language 1

\def\rep#1{\mbox{$\langle#1\rangle$}}
\def\qed{{\hfill{\vrule height5pt width3pt depth0pt}\medskip}}


\def\rep#1{\mbox{$\langle#1\rangle$}}
\newfont{\bbc}{msbm10}
\def\Bbb#1{\hbox{{\bbc #1}}}
\newcommand{\inte }{{\rm int}\,}
\newcommand{\diam }{\,{\rm diam}\,}
\newcommand{\cl }{{\rm cl}\,}
\newcommand{\sgn }{{\rm sgn}\,}
\newcommand{\bd }{\partial}
\newcommand{\im }{\mathrm{Im} \,}
\newcommand{\ind }{\mathrm{ind} \,}
\newcommand{\dom }{\mathrm{dom} \,}
\newcommand{\orbit}[1]{\mathcal O\left(#1\right)}
\newcommand{\Fix}{\mathrm{Fix}}
\newcommand{\cover}[1]{\stackrel{#1}{\Longrightarrow}}
\newcommand{\lcover}[1]{\stackrel{#1}{{=\Longrightarrow}}}
\newcommand{\coverOned}[1]{\stackrel{#1}{\rightarrow}}

\newtheorem{theorem}{Theorem}
\newtheorem{lemma}[theorem]{Lemma}
\newtheorem{remark}[theorem]{Remark}
\newtheorem{definition}[theorem]{Definicja}

\hyphenation{
    mul-ti-variate
    semi-conju-gated
    com-pli-ca-ted
    dy-na-mics
    he-te-ro-cli-nic
}


\begin{document}

\title{Magisterka Maciej Zgliczyński}

\input Introduction.tex
\input Notation.tex
\input BrouwerDegree.tex
\input Definitions.tex
\input Dynamics.tex
\input numeric.tex
\input Application.tex
\input MyStuff.tex

\section{Kod}


\begin{thebibliography}{GSS}

\bibitem[N]{N} L. Nirenberg, Topics in Nonlinear Functional Analisys, New York University  1974
\bibitem[Sch]{Sch} J. Schwartz, Nonlinear Functional Analisys, Gordon and Breach ,New York 1969 
\bibitem[AM]{AM} Z. Arai and K. Mischaikow, \emph{Rigorous Computations of Homoclinic Tangencies}, SIAM J. on Appl. Dyn. Sys. 5 (2006), 280--292.


\bibitem[AZ]{AZ} G. Arioli and P. Zgliczy\'nski, \emph{Periodic, homoclinic and heteroclinic orbits
for H\'{e}non Heiles Hamiltonian near the critical energy level},
Nonlinearity, vol. 16, No. 5 (2003), 1833--1852

\bibitem[CAPD]{CAPD}{\rm  CAPD -- Computer Assisted Proofs in Dynamics
group}, a C++ package for rigorous numerics, {\tt
http://capd.wsb-nlu.edu.pl.}

\bibitem[D]{D} R.I. Devaney, \emph{A first course in chaotic dynamical systems: Theory and
experiments}, Addison-Wesley, 1992.

\bibitem[GZ]{GZ} Z. Galias and P. Zgliczy\'nski, \emph{Abundance of homoclinic and heteroclinic orbits and
rigorous bounds for the topological entropy for the H\'enon map},
Nonlinearity, 14 (2001) 909--932.

\bibitem[IE]{IE} \emph{The IEEE Standard for Binary Floating-Point
Arithmetics}, ANSI-IEEE Std 754, (1985).

\bibitem[KWZ]{KWZ} H. Kokubu, D.Wilczak,  P. Zgliczy\'nski,
 \emph{Rigorous verification of cocoon bifurcations in the Michelson
system}, Nonlinearity, 20 (2007), 2147-2174.

\bibitem[Li]{Li} Q. Li, \emph{A topological horseshoe in the hyperchaotic R\"ossler attractor},
Physics Letters A, 372 (2008) 2989--2994.
\bibitem[L]{L} N. G. Lloyd, \emph{Degree theory}, Cambridge
Tracts in Math., No. 73, Cambridge Univ. Press, London, 1978

\bibitem[MM]{MM} K. Mischaikow and M. Mrozek,
\emph{Chaos in Lorenz equations: a computer assisted proof},
Bull. Amer. Math. Soc. (N.S.), 33(1995), 66-72.

\bibitem[Mo]{Mo} R.E. Moore, \emph{Interval Analysis.} Prentice
Hall, Englewood Cliffs, N.J., 1966


\bibitem[P03]{P03} P. Pilarczyk, \emph{Topological numerical approach to the existence of periodic trajectories in ODE's},
Discrete and Continuous Dynamical Systems, A Supplement Volume:
Dynamical Systems and Differential Equations, 701-708 (2003).


\bibitem[R76]{R76} O.E. R\"ossler, \emph{An Equation for Continuous Chaos}, Physics Letters, Vol. 57A no 5, pp 397--398, 1976.

\bibitem[R79]{R79} O.E. R\"ossler, \emph{An equation for hyperchaos}, Physics Letters A 71 , no.2-3, (1979), 155--157.

\bibitem[SK]{SK} D. Stoffer and U. Kirchgraber, \emph{Possible chaotic motion of comets in the Sun Jupiter system -
an efficient computer-assisted approach}, Nonlinearity, 17 (2004)
281-300.

\bibitem[T]{T} W. Tucker, \emph{A Rigorous ODE solver and Smale's 14th Problem},
Found. Comput. Math., 2:1, 53-117, 2002.

\bibitem[W]{W} D. Wilczak, {\tt
http://www.ii.uj.edu.pl/\~{}wilczak.}

\bibitem[WZ]{WZ} D. Wilczak and P. Zgliczy\'nski, \emph{Heteroclinic Connections between Periodic Orbits in
Planar Restricted Circular Three Body Problem - A Computer Assisted
Proof}, Comm. Math. Phys.,  234 (2003) 1, 37-75.

\bibitem[WZ2]{WZ2} D. Wilczak and P. Zgliczy\'nski, \emph{Topological method
for symmetric periodic orbits for maps with a reversing symmetry},
 Discrete Cont. Dyn. Sys. A 17, 629--652 (2007).

\bibitem[WZ3]{WZ3} D. Wilczak and P. Zgliczy\'nski, \emph{Period doubling in the R\"ossler system - a computer assisted proof},
 Foundations of Computational Mathematics, to appear.

\bibitem[ZGi]{ZGi} P. Zgliczy\'nski and  M. Gidea, \emph{Covering relations for multidimensional
     dynamical systems},  J.  Differential Equations, 202/1(2004), 33--58

\bibitem[Z]{Z} P. Zgliczy\'nski, \emph{Computer assisted proof of
chaos in the R\"ossler equations and the H\'enon map},  Nonlinearity {\bf 10}
(1997), 243-252.

\bibitem[Z1]{Z1} P. Zgliczy\'nski,  \emph{$C^1$-Lohner algorithm}, Foundations
of Computational Mathematics, {\bf 2} (2002), 429--465.

\end{thebibliography}

\end{document}
