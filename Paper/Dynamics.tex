\title{Układy dynamiczne}

Tu trzeba jakiś wstęp 

\begin{definition}
 \label{ukladDynamiczny}
 Parę (X,$\varphi$) nazywamy układem dynamicznym na $X$ jeżeli dla wszystkich  $ x \in X $  oraz $t,s \in \mathbb R $ spełnione są następujące warunki
    $$  \varphi(x,0) = x$$
    $$ \varphi(\varphi(x,t),s) = \varphi(x,t+2) $$



    

%$$


\end{definition}

    
Rozważmy równanie różniczkowe postaci:
	$$ \dot{x} = f(x) $$
Gdzie $ f : \mathbb R^n \to \mathbb R^n $ jest funkcją Lipchitzowską ( np jest różniczkowalna klasy $C^1$) . Na mocy twierdzenia Picarda istnieje dokładnie jedno rozwiązanie problemu Cauchy'egp
  $$ \dot{x} = f(x(t)) $$
  $$ x(0) = x_0 $$

Z tego twierdzenia wynika również, że rozwiązanie jednoznacznie przedłuża się do rozwiązania wysyconego :
 $$ I_x \to \mathbb R^n $$ 
 Gdzie $ I_x$ jest maksymalnym przedziałem istnienia rozwiązania. Jeżeli dla każdego  $x \in \mathbb R^n $ 
 $ I_x = \mathbb R $ to równanie  $ \dot{x} $ indukuje układ dynamiczny:
  $$
    \varphi(x_0,t) := x(t)
  $$
  
  Gdzie $x(t) $ jest rozwiązaniem problemu Cauchy'ego.
  Często zdarza się, że rozwiązania wysycone nie są określone dla wszystkich czasów na osi liczb rzeczywistych, czyli $ I_x \neq \mathbb R $ dla 
  pewnych $x \in \mathbb R^n $ . W takim przypadku dane równanie różniczkowe nie indukuje (pełnego) układu dynamicznego tylko 
  lokalny układ dynamiczny.

 \begin{definition}
  \label{lokalnyUkladDynamiczny}
  
    Niech X będzie przestrzenią topologiczną $ \Omega \subset X \times \mathbb R$ podzbiorem takim , że  $ X \times {0} \subset \Omega $ oraz 
    $ \varphi : \Omega \to X $ będzie ciągłe. Dla  $ x \in X $ określamy 
	$$ I_x := \{ t \in \mathbb R | (x,t) \in \Omega \}$$
    Trójkę $(X,\Omega,\varphi) $ nazywamy lokalnym układem dynamicznym, jeżeli są spełnione następujące warunki
    
    \begin{enumerate}
     \item $\Omega $ jest podzbiorem otwartym $ X\times \mathbb R $ oraz dla wszystkich $ x \in X , I_x $ jest przedziałem otartym
     \item dla każdego $x\in X$
	$$ \varphi(x,0) = x $$
     \item jeżeli $x \in X $, $ t \in I_x$ oraz $ s\in I_{\varphi(x,t},$ to $ t+s \in I_x $ oraz $ \varphi(\varphi(x,t),s) = \varphi(x,t+s) $
     \item jeżeli $ t\in I_x $ to $-t \in I_{\varphi(x,t} $
    \end{enumerate}

 \end{definition}

 Oczywiście każdy układ dynamiczny jest lokalnym układem dynamicznym. 
 Podobne definicje można sformułować dla układów z czasem dyskretnym. Mówimy wtedy o dyskretnych układach dynamicznych bądź dyskretnych lokalnych układach dynamicznych
 
 \begin{definition}
 \label{iterowanyUkladDynamiczny}
 
 
  Niech X będzie przestrzenią topologiczną $ \varphi : X \times \mathbb{Z} \to X $ będzie ciągłe. Parę $ (X,\varphi) $ nazywamy dyskretnym układem dynamicznym,
  jeżeli dla $ x \in X $ oraz $ m,n \in \mathbb Z $ są spełnione następujące warunki :
    $$
      \varphi(x,0) = x \quad
      \varphi(\varphi(x,m),n) = \varphi(x,m+n)
    $$
    
 \end{definition}
 
 Dyskretne układy przeważnie definiuje się jako iteracje pewnego homeomorfizmu $ f : X \to X $.
 
  $$
   \varphi(x,n) := f^n(x)
  $$
  
  W takiej sytuacji będziemy parę $(X,f)$ będziemy nazywać dyskretnym układem dynamicznym
  
  Analogicznie do przypadku ciągłych układów dynamicznych definiujemy pojęcie lokalnego dyskretnego 
  układu dynamicznego.
  
  
  \begin{definition}
    Niech $ X $ będzie przestrzenią topologiczną, $\Omega \subset X \times \mathbb Z $ podzbiorem takim, że $ X\times \{0\} \subset \Omega $ 
    oraz $ \varphi : \Omega \to X $ będzie ciągłe. Dla $ x \in X $ definiujemy 
    $$
	I_x := \{n \in \mathbb Z | (x,n) \in \Omega \}
    $$
    
    Trójkę $(X,\Omega,\varphi) $ nazywamy lokalnym dyskretnym układem dynamicznym, jeżeli są spełnione następujące warunki
    
    \begin{enumerate}
     \item dla każdego $ x\in X , I_x $ jest przedziałem w $ \mathbb Z $
     \item dla każdego $x\in X$, 
	    $$ \varphi(x,0) = x $$
     \item jeżeli $x \in X $, $ m \in I_x$ oraz $ n\in I_{\varphi(x,m},$ to $ m+n \in I_x $ oraz $ \varphi(\varphi(x,t),n) = \varphi(x,m+n) $
     \item jeżeli $ m\in I_x $ to $-m \in I_{\varphi(x,t)} $
    \end{enumerate}
    
    
    
  \end{definition}
  
  Dla pary $(X,f) $ gdzie $X$ to przestrzeń topologiczna a $ f $ to homeomorfizm na obraz można w naturnalny spodób wprowadzić 
  lokalny dyskretny układ dynamiczny. Definiując go mianowicie :
  $$
        \varphi(x,0) = x  $$ 
        $$
         \forall n \in \mathbb Z , x \in dom(f^n) \varphi(x,n) = f^n(x)
  $$
  
  Jeżeli to będzie jasne z kontekstu będziemy korzystali z takiego oznaczania, tzn będziemy parę $(X,f)$ nazywali 
  (lokalnym) dyskretnym układem dynamicznym skonstruowanym w powyższy sposób.

  \begin{definition}
   Niech $(X,\Omega,\varphi) $ będzie lokalnym (dyskretnym) układem dynamicznym. Dla punktu $ x \in X $ definiujemy orbitę punktu
   $ x $ jako 
   $$
	\orbit x := \{ y \in X |  y = \varphi(x,t) , t \in I_x\}
   $$
  \end{definition}
  
  \section{Odwzorowanie Poincar\'ego}
  
  Bardzo często przy badaniu równań różniczkowych i związanych z nimi lokalnymi układami dynamicznym jest dyskretyzacja czasu.
  Próbujemy jakoś z naszego lokalnego układu dynamicznego stworzyć lokalny dyskretny układ dynamiczny. Istnieją dwie sensowne, wykorzystywane
  techniki. Pierwsza z nich to ustalenie kroku czasowego $ h \in \mathbb R $ i badanie przesunięć punktów po danym czasie w naszym układzie dynamicznym
  $$
    \varphi_t : X \oarrow X,  \varphi_t(x) := \varphi(x,t)
  $$.
  
  Drugim podejściem jest odwzorowanie Poincar\'ego zdefiniowane następująco
  
  \begin{definition}
    Niech $\dot(x) = f(x)$ będzie równaniem różniczkowym z prawą stroną gładką, $ f : \mathbb R^n \to \mathbb R^n$. Podzbiór $ \Theta \subset \mathbb R^n $ 
    będziemy nazywać lokalną sekcją równania $ \dot(x) = f(x) $ , jeżeli
    
    \begin{enumerate}
     \item $ \Theta $ jest rozmaitością wymiaru n-1 bez brzegu
     \item dla każdego $ x \in \Theta$ pole wektorowe jest transwersalne do sekcji w $x$.
     Mianowicie, iloczyn skalarny $n(x) \cdot f(x) \neq 0 $ gdzie $n(x)$ oznacza wektor normalny do $ \Theta $ w punkcie x
    \end{enumerate}

  \end{definition}
  
  Ponieważ w każdym puncie lokalnej sekcji $ \Theta $ pole wektorowe jest do niej transwersalne (warunek 2), to każde 
  rozwiązanie (istnienie i jednoznaczność mamy z twierdzenia Cauchy-ego Picarda ) z warunkiem początkowym 
  $ x(0) = x_0 \in \Theta $ musi opuścić sekcję $ \Theta $, czyli:
      $$
       \forall x \in \Theta \exists{\epsilon > 0} \forall t \in (0,\epsilon) \varphi(x_0,t) \notin \Theta 
      $$
 
  Może sie zdarzyć, że pewne punkty z lokalnej sekcji wrócą po pewnym czasie do sekcji $ \Theta $ po pewnym
  czasie $ t > $, czyli że orbita przetnie ponownie lokalną sekcje po czasie $ t > 0 $ . 
  Warunek ten definiuje odwzorowanie Poincar\'ego. Określamy:
  $$
    T_x := \{ t > 0 | \varphi(x,t) \in \Theta \}.
  $$

  \begin{definition} Niech $ \Theta $ będzie lokalną sekcją równania $ \dot(x) = f(x)$. Wtedy odwzorowanie 
    $ P: \Theta \oarrow \Theta $ określone następująco :
    \begin{enumerate}
     \item $ x \in dom(P) \Leftrightarrow T_x \neq 0 $
     \item $ P(x) := \varphi(x,t_x)$, gdzie $t_x := min{t \in T_x} $
    \end{enumerate}
    Nazywamy odwzorowaniem Poincar\'ego

  \end{definition}
  
Odwzorowanie Poincar\'ego jest często wykorzystywane do szukania rozwiązań okresowych, heteroklicznych lub homokliniczynch wyjściowego
równania różniczkowego. Na przykład by udowodnić, że istnieje rozwiązanie okresowe dla okresowe wystarczy pokazać, że pewne odwzorowanie 
Poincar\'ego ma punkt stały lub okresowy. Do tego już możemy wykorzystać znane narzędzia topologiczne jak chociażby lokalny stopień Brouwera.


\section{Symetrie}

Wiele układów dynamicznych posiada pewne symetrie lub symetrie odwracania czasu.
\begin{definition}
  Homeomorfizm $ S : X \to X $ będziemy nazywać symetrią lokalnego (dyskretnego ) układu dynamicznego $(X,\Omega,\varphi) $ jeżeli ]
  dla wszystkich $ x \in X $ oraz $ t \in I_x $ zachodzą następujące warunki :
  \begin{enumerate}
   \item $ t \in I_{S(x)}$,
   \item $ S(\varphi(x,t)) = \varphi(S(x),t) $.
  \end{enumerate}

\end{definition}


\begin{definition}
  Homeomorfizm $ R: X \to X $ nazywamy symetrią odwrócenia czasu w lokalnym (dyskretnym) układzie dynamicznym $ (X,\Omega,\varphi)$,
  jeżeli dla wszystkich $ x \in X $ oraz $ t \in I_x $ zachodzą warunki 
  \begin{enumerate}
   \item $ -t \in I_{R(x)}$,
   \item $ R(\varphi(x,t)) = \varphi(R(x), -t)$
  \end{enumerate}

\end{definition}

Dla układów dyskretnych $(X,f) $ definicje tą można zapisać równoważnie jako :
  $$
   \forall n \in \mathbb Z, \forall x \in dom(f^n) R(x) \in dom(f^{-n}), f^{-n}(R(x)) = R(f^n(x))
  $$

\begin{lemma}
  \label{lem:reversedTime}
   Niech (X,f) będzie lokalnym dyskretnym układem dynamicznym, $ R : X\to X $ pewnym homeomorfizmem. R jest 
   symetrią odwrócenia w czasu dla f wtedy i tylko wtedy, gdy dla wszystkich $ x \in dom(f) $ zachodzi
   \begin{equation}
       R(x) \in dom(f^{-1}) \mbox{ oraz } \quad R(f(x)) = f^{-1}(R(x)).
   \end{equation}

\end{lemma}

\textbf{Dowód:}
  Zauważmy, że gdy $R$ jest symetrią odwrócenia czasu to warunek z lematu zachodzi. Mianowicie z definicji
  symetrii odwrócenia dla $ x \in dom(f) $ zachodzi 
    \begin{equation}
	R(f(x)) = R(\varphi(x,1)) = \varphi(R(x),1) = f^{-1}(R(x))
    \end{equation}

    W ten sposób otrzymujemy implikacje w jedną stronę. Należy teraz udowodnić, że jeżeli dla funkcji $ R $ 
    zachodzą warunki opisane w tezie lematu to rzeczywiście jest ona symetrią odwrócenia w czasie.
    Mianowice wystarczy wykazać , że $ \forall x \in dom(f^n) $
    \begin{equation}
      \label{eq:odwrteza}
       R(x) \in dom(f^{-n}) \mbox{ oraz } R(f^n(x)) = f^{-n}(R(x))
    \end{equation}

    Dowód przeprowadzimy w dwóch etapach. Najpierw indukcyjnie dla $ n > 0 $ a następnie dla $ n < 0 $.
    Dla $ n = 0 $ warunek jest trywialnie spełniony ponieważ $ f^0 := Id $.
    Dla $ n = 1 $ warunek $ R(f(x)) = f^{-1}(R(x)) $ jest konsekwencją założenia implikacji.
    Załóżmy teraz, że \eqref{eq:odwrteza} zachodzi dla wszystkich $ 1 \leq k \leq n $. Chcemy wykazać, że 
    zachodzi również dla $ n+1$.
    Niech $ x \in dom(f^{n+1}) $ wtedy mamy $ x \in dom(f) $ oraz $ f(x) \in dom(f^n) $. Z założenia 
    implikacji otrzymujemy, że $ R(x) \in dom(f^{-1})$ a z założenia indukcyjnego 
      $$
	f^{-1}(R(x)) = R(f(x)) \in dom(f^{-n})
      $$
      
      Wynika stąd, że $ R(x) \in dom(f^{-(n+1)}) $ oraz
     \begin{eqnarray*}
	  R(f^{n+1}(x))=R(f^n(f(x)))\stackrel{\mbox{\small z zał
	  ind.}}=\\f^{-n}(R(f(x)))=f^{-n}(f^{-1}(R(x)))=f^{-(n+1)}(R(x)).
      \end{eqnarray*}
    Czyli udowodniliśmy krok indukcyjny.
    Teraz będziemy chcieli wykazać tezę dla $ u < 0 $ czyli chcemy pokazać,
    że jeżeli $ x \in  dom(f^{-n} $ gdzie $ n \in \mathbb N$ to 
	\begin{equation}
	  R(x) \in dom(f^n) \quad \mbox{ oraz } R(f^{-n}(x)) = f^{n}(R(x))
	\end{equation}
	
    Niech $ x \in dom(f^{-n}) $ z tego wynika, że $ f^{-n}(x) \in dom(f^n) $
    Teraz z \eqref{eq:odwrteza} otrzymujemy :
    $$
      R(f^{-n}(x)) \in dom(f^{-n}) \mbox{ oraz } R(f^{n}(f^{-n}(x))) = f^{-n}(R(f^{-n}(x)))
    $$
    
    Z tego wynika, że
    $$
      R(x) = f^{-n}(R(f^{-n}(x))) \in dom(f^n), \quad f^n(R(x)) = R(f^{-n}(x))
    $$
    
    co kończy dowód lematu.
    
Często układy dynamiczne, które posiadaj symetrie lub symetrie odwrócenia czasu posiadają również punkty
których orbity są nieimiennicze względem symetrii. Takie orbity będziemy nazywać symetrycznymi. 

\begin{definition}
  Niech $(X,\Omega,\varphi) $ będzie lokalnym (dyskretnym) układem dynamicznym a H pewną symetrią ( odwrócenia czasu) tego układu.
  Orbitę punktu $ x\in X $ będziemy nazywać symetryczną, jeżeli $ H(\orbit(x) = \orbit(x). $
  
\end{definition}

Przy pewnych założeniach o sekcji Poincar\'ego. Symetrie dla całego układu implikuje symetrie dla odwzorowania Poincar'ego.
Dokładny sformułowanie tego spostrzeżenia zamieszczam poniżej w formie lematu.

\begin{lemma}
  \label{lem:poincareReverse}
  Niech $(X,\Omega,\varphi) $ będzie lokalnym układem dynamicznym indukowanym przez pewne równanie różniczkowe, oraz R będzie
  symetrią odwrócenia czasu dla $ \varphi $. Załóżmy, że $ \Theta \subset X $ jest lokalną sekcją dla $\varphi$ taką,że
  $ R(\Theta) = \Theta$. Wtedy $R_{| \Theta} $ jest symetrią odwrócenia czasu dla odwzorowania Poincar\'ego $ P: \Theta \oarrow \Theta$.
\end{lemma}

\textbf{Dowód:}
Na podstawie wcześniej wykazanego lematu \eqref{lem:reversedTime} wystarczy pokazać, że dla $ x\in dom(P) $ 
zachodzi
$$
  R(x) \in dom(P^{-1}) \mbox{ oraz } R(P(x)) = P^{-1}(R(x))
$$

Niech $ x \in dom(P)$. Z definicji odwzorowania Poincar\'ego istnieje $ t_x > 0 $ takie, że $ P(x) = \varphi(x,t_x)$.
Chcemy pokazać, że $ R(x) \in dom(P^{-1}) $ oraz $ P^{-1}(R(x)) = R(P(x)) $, dokładniej :
\begin{enumerate}
 \item $\varphi(R(x),-t_x) \in \Theta $
 \item dla $ 0 < t < t_x, \quad \varphi(R(x),-t) \notin \Theta$.
\end{enumerate}
Dowód pierwszego punktu wynika z z faktu, że R jest symetrią odwrócenia czasu w układzie ciągłym czyli skoro 
$ t_x \in I_x $ to z tego wynika, że  $ -t_x \in I_{R(x)} $ i 
$$
  \varphi(R(x),-t_x) = R(\varphi(x,t_x)) = R(P(x)) \in R(\Theta)= \Theta
$$
Stąd wynika, że $ R(x) \in dom(P^{-1})$ 
Drugi punkt będziemy dowodzić nie-wprost. Załóżmy, że nie zachodzi tzn, istnieje pewien 
$ t \in (0,t_x) $ taki, że $ \varphi(R(x),-t) \in \Theta $. Wtedy z założenia, że R
jest symetrią odwracania czasu dla naszego układu otrzymujemy :
 $$
  R(\varphi(x,t)) = \varphi(R(x), -t) \in \Theta
 $$ 
 R jest homeomorfizmem i $ R(\Theta) = \Theta $ a więc również $ R^{-1}(\Theta) = \Theta $,
 a wtedy:
 $$
    \varphi(x,t) = R^{-1}(R(\varphi(x,t))) = R^{-1}(\varphi(R(x),-t)) \in R^{-1}(\Theta) = \Theta.
 $$

 Czyli otrzymaliśmy sprzeczność z wyborem $t_x$,które była zdefiniowane jako minimum czasów w których orbita
 $ x $ przecina $\Theta$, w ten sposób udowodniliśmy lemat.
 