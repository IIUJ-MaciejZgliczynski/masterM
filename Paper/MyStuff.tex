
\section{Dowód numeryczny}

\subsection{Reprezentacja h-setów}

W tej pracy mamy jedynie doczynienia z {\em h-setami} na płaszczyźnie które mają dokładnie jeden kierunek 
nominalnie stabilny i niestabilny. Dlatego będziemy reprezentować nasze h-sety  w jako następujące trójki 
$ (x,s,u) $, gdzie $x,s,u \in \mathbb R^2 $ i $s,u$ są liniowo niezależne. wtedy definiujemy 
$$
  |N| := \{v \in \mathbb R^2 | \exists_{t_1,t_2 \in [-1,1]} \quad v = x + t_1 \cdot s + t_2 \cdot u  \}
      = x + [-1,1]\cdot s + [-1,1]\cdot u 
$$

Homeomorfizm $ c_N $ jest odwzorowaniem afinicznym zdefiniowanym w następujący sposób 
$$
  c_N(v) = A^{-1}(v - x) 
$$

Gdzie $ A = [u,s] $ jest macierzą kwadratową. Przy takiej reprezentacji $h-setu$ zachodzą następujące równości

\begin{eqnarray}
  N^- & = & x + \{-1,1\} \cdot s \\
  N^+ & = & x + u \cdot \{-1,1\}  
\end{eqnarray}

W takiej sytuacji będziemy o oznaczać $ N = h(x,u,s) $

\subsection{Dowodzenie relacji nakrywającej}

Niech $ M $ będzie $ h-setami$ o jednym kierunku stabilnym i jednym kierunku niestabilnym wtedy 
$ \mathbb{R}^2 \setminus |M| $ w naturalny 
sposób rozbija się na cztery części. Mianowice:

\begin{eqnarray}
  M^T & = & \{ x \in R^2 | \exists_{ t_1 \in (1,\inf), \quad t_2 \in \mathbb R} : x =  \cdot s + t_2 \cdot u  \} \\
  M^B & = & \{ x \in R^2 | \exists_{ t_1 \in (-\inf, -1), \quad t_2 \in \mathbb R} : x = t_1 \cdot s + t_2 \cdot u  \} \\
  M^L & = & \{ x \in R^2 | \exists_{ t_1 \in \mathbb R, \quad t_2 \in (-\inf,-1)} : x = t_1 \cdot s + t_2 \cdot u  \} \\
  M^R & = & \{ x \in R^2 | \exists_{ t_1 \in \mathbb R, \quad t_2 \in (1,\inf)} : x = t_1 \cdot s + t_2 \cdot u  \} \\
\end{eqnarray}


Mając dane $ h-sety $ $ N,M $ i funkcje $ f : \mathbb R^2 \to \mathbb R^2 $ będziemy chcieli wykazać, że  $N \cover{f} M$ , wystarczy 
wykazać, że zachodzą następujące warunki :
\begin{enumerate}
 \item $ f_c(N_c) \cap ( \overline{M^T \cup M^B} ) = \emptyset $
 \item $ f_c(N^-_c) = N^{-}_1 \cup N^{-}_2 $, $ N^{-}_1 \cap N^{-}_2 = \emptyset $ i $ N^-_1 \subset N^L , N^-_2 \subset N^R $ 
\end{enumerate}

Oczywiście jeżeli $ N \cover{f} M $ to powyższe dwa warunki są spełnione.
Pierwszy warunek wynika z faktu, że mając $h_c $ homotopie z definicji relacji nakrywającej z warunku 
$ h([0,1],N_c) \cap M^+_c = \emptyset $ wynika pierwszy warunek. Z warunku $ h([0,1],N^-_c) \cap M_c = \emptyset $ i z faktu, że nasze 
odwzorowanie jest homotopijne z rzutowaniem, którego stopień po zawężeniu dziedziny do obrazu jest niezerowy wynika drugi warunku.

Implikacja w drugą stronę też nie jest trudna. Mianowice zakładając, że zachodzą powyższe dwa warunki można wskazać odpowiednią homotopię,
która spełni warunki z relacji nakrywającej.
Mianowicie :

$$
  h(t,x) = t( \cdot p \cdot \pi_2(x),0) + (1-t)\cdot f(x)
$$

gdzie $ p \in \{-2,2\} $ w zależności od zachowanie się funkcji $ f $ na $ N^-$. Jeśli $ \pi_2(f_c( -1,-1)) < -1 $ to wtedy bierzemy $ p = -2 $ 
a jeżeli $ \pi_2(f_c( -1,-1)) > 1 $  to bierzemy $ p = 2 $ . 
Tak zdefiniowana homotopia spełnia warunki z definicji relacji nakrywającej, ponieważ:
\begin{enumerate}
 \item $ h_0(x) = h(0,x) = 0 \cdot p \cdot x + (1-0) \cdot f(x) = f(x) $
 \item Niech $ x \in N^-_c $, wtedy z wyboru p wynika, że oba elementy których bierzemy kombinacje wypukłą leżą albo w $ M^L $ albo w $ M^R$ 
 te dwa zbiory są wypukłe a więc ich kombinacja również należy do tego zbioru. Oczywiście są one rozłączne z $ M_c$.
 \item Ponieważ $ \mathbb R^2 \ (M^T \cup M^B) $ jest zbiorem wypukłym i oba element których kombinacje wypukłą rozważamy należą 
 zbioru to ich kombinacja również.
 \item Drugi punkt definicji relacji nakrywającej jest również spełniony w trywialny sposób ponieważ nasze liniowe 
 odwzorowanie to jest po prostu
 $ x \to p \cdot x $ gdzie $ p $ = 2 lub -2. Czyli ma niezerowy stopień. Mianowicie 1 lub -1.
\end{enumerate}

Czyli wszystkie warunki dla relacji nakrywającej są spełnione, a więc $N \cover{f,w} M $. Gdzie $ w \in \{-1,1\} $ 
W ten sposób otrzymaliśmy łatwy sposób weryfikowania czy zachodzi dana relacja nakrywająca czy nie. Zauważmy, że powyższe dwa 
warunki można wyrazić za pomocą odpowiednich nierówności. A więc możemy sprawdzić je na komputerze wykorzystując arytmetykę przedziałową.

\subsection{Kluczowy dowód}

\begin{theorem}
  Dla każdego parametru $ c = 0.49 $ układ Michelsona jest $ \Sigma_4 $ chaotyczny,
  czyli, że istnieje odwzorowanie Poincar\'ego takie, że jest pół sprzężone z pełnym shiftem na czterech symbolach.
\end{theorem}

Dowód:

By wykazać następujące twierdzenie będziemy badać odwzorowanie Poincar\'ego układu Michelsona z sekcją $ \theta $, zdefiniowaną wcześniej.
Dla skrócenia zapisu będziemy utożsamiać naszą sekcje $ \theta $ z $ \mathbb R^2 $.
Rozważmy następujące $ h-sety $ $ N_i = h(x_i,u_i,s_i) $ dla $ i \in 1 \cdots 5 $ :

\begin{eqnarray*}
 x_1 & = &(0,0.88930181632132043745) \\  u_1 & = & (-0.090000000000000010547,-0.036352256923916359543) \\
 s_1 & = & (0.090000000000000010547,-0.036352256923916359543) \\
 x_2 & = & (0,-1.1688058965368766096) \\ u_2 & = &(-0.045478496673335325196,-0.090000000000000010547) \\
 s_2 & = &(0.041769254863549584722,-0.090000000000000010547) \\
 x_3 & = & (0,-0.21856378422848130039)\\  u_3 & = & (0.10799999999999999878,-0.038322644233375995071) \\
 s_3 & = &(-0.10799999999999999878,-0.038322644233375995071) \\
 x_4 & = & (0.61254887767896715189,0.655149395511700261) \\ u_4 & = & (-0.04950000000000000927,-0.022192443886905721673) \\
 s_4 & = & (0.13050000000000000488,-0.057865711545585452047) \\
 x_5 & = & (-0.61254887767896715189,0.655149395511700261)  \\ u_5 & = & (-0.13050000000000000488,-0.057865711545585452047) \\
 s_5 & = & (0.04950000000000000927,-0.022192443886905721673) \\
\end{eqnarray*}

Niech $ N = \bigcup^5_{i=1} | N_i| $. Niech $ P : \theta \oarrow \theta $ oznacza odwzorowanie Poincare\'go dla 
układu Michelsona z parametrem $ c = 0.49 $

\begin{lemma}
\label{num:cov}
 Funkcja częściowa P jest określona na całej swojej dziedzinie.
 Co więcej zachodzą dwa następujące ciągi relacji nakrywających. Mianowicie :
 \begin{eqnarray*}
    N_1 \cover{P} N_2 \cover{P} N_1 \\
    N_2 \cover{P} N_5 \cover{P} N_3 \cover{P} N_4 \cover{P} N_2 
 \end{eqnarray*}
 
 Dowód komputerowy tego lematu, można przeprowadzić za pomocą załączonego oprogramowania do tej pracy.
 Warto zauważyć, że te dwie relacje nakrywające są istotnie różne ale mają jeden element wspólny mianowicie $ N_2 $.
 

\end{lemma}
Uwaga do dowodu numerycznego. Udowodniamy, wszystkie relacje nakrywające 
Korzystając z tego lematu możemy przejść do dowodu kluczowego twierdzenie w tej pracy.
Dowód:
Zdefiniujmy $ \Sigma_2 = \{ 2,5 \} $ i odwzorowanie shift 
$$
  \sigma : \Sigma_2 \ni (x_i)_{i \ in \mathbb Z} \to (x_{i+1})_{i \in \mathbb Z } \in \Sigma_2
$$

Niech $ p : | N_2 | \cup | N_5 | \oarrow \Sigma2 $ będzie zdefiniowana w następujący sposób:
\begin{eqnarray*}
  dom(p) := \{ x \Theta | \forall j \in \mathbb Z, x \in dom(P^{4j}), P^{4j}(x) \in |N_2| \cup |N_5| \} \\ 
  p(x) = (i_j) j \in \mathbb Z \Leftrightarrow P^{4j}(x) \in |N_{i_j}, \text{ dla } x \in dom(p)
\end{eqnarray*}

Ponieważ zbiory $ N_2, N_5 $ są rozłączne definicją naszego odwzorowania jest poprawna. Co więcej dla $ x \in dom(p) $ zachodzi :
$$
  p(P^4(x)) = \sigma(p(x))
$$
.
Z relacji nakrywających wykazanych w \ref{num:cov}. Możemy wywnioskować, że zachodzą następujące relacje nakrywające dla odwzorowania $ P $
\begin{eqnarray*}
  N_2 \cover{P} N_1 \cover{P} N_2 \cover{P} N_1 \cover{P} N_5 \\
  N_2 \cover{P} N_1 \cover{P} N_2 \cover{P} N_1 \cover{P} N_2 \\
  N_5 \cover{P} N_3 \cover{P} N_4 \cover{P} N_1 \cover{P} N_2 \\
  N_5 \cover{P} N_3 \cover{P} N_4 \cover{P} N_2 \cover{P} N_5
\end{eqnarray*}

  Z tych relacji otrzymujemy, że 
\begin{eqnarray*}
  N_2 \cover{ P^4 } N_5 \\
  N_2 \cover{ P^4 } N_2 \\
  N_5 \cover{ P^4 } N_2 \\
  N_5 \cover{ P^4 } N_5 
\end{eqnarray*}

Z tych relacji i z twierdzenia \ref{th:top} otrzymujemy, że odwzorowanie $ p $ jest dobrze określone. Mianowice dowolny $ x \in \Sigma_2 $ 
jest realizowany jako pełna orbita. Z twierdzenia \ref{th:top} wynika, że dla każdej pełnej orbity istnieje punkt, który ją realizuje. Co więcej
jeżeli element $ x \in \Sigma_2 $ jest okresowy, to istnieje punkt w $ p^{-1} $ który jest okresowy. 
W ten sposób wykazaliśmy, że czwarta iteracja naszego odwzorowania Poincare\' jest pół sprzężona z pełnym shiftem na dwóch symbolach.














