\title{Notacja}

\section{Notacja}
Przy ustalonej normie w $ \mathbb{R}^n $ poprzez $B_n(c,r)$ będziemy 
oznaczali otwartą kule o promieniu $r$ i środku w $ c \in \mathbb{R}^n $.
Jeżeli wymiar $n$ będzie jasny z kontektu to będziemy opuszczać indeks $n$.
Niech  $S^n(c,r)=\partial B_{n+1}(c,r)$, poprzez $S^n$ będziemy oznaczać $S^n(0,1)$.
Dla $n=0$ defniujemy odpowiednio $\mathbb{R}^0=\{0\}$, $B_0(0,r)=\{0\}$, $\partial B_0(0,r)=\emptyset$.


Przy ustalonym $Z$, poprzez $\inte Z$ , $\overline{Z}$, $\partial Z$ 
oznaczamy odpowiednio wnętrze, domknięcie i brzeg {\em h-setu} $Z$.
Dla odwzorowania $h:[0,1]\times Z \to \mathbb{R}^n$ ustalamy  $h_t=h(t,\cdot)$.
Poprzez $\mbox{Id}$ oznacamy odzworowanie identycznościowe. 
Poprzez $f:X\oarrow Y$ ozaczamy odwzorwanie częściowe, czyli 
odwzorowanie którego dziedziną niekoniecznie jest całe $X$. 
Dla odwzorwanie $f:X\oarrow Y$, poprzez $\mbox{dom}(f)$ będziemy
oznaczać dziedzine $f$. Dla $N \subset \Omega$, $N$-otwarty i $ c \in 
\mathbb{R}^n$ . Poprzez $ det A $ gdzie $ A $ to macierz kwadratowa będziemy 
oznaczać wyznacznik macierzy $ A $.
Poprzez $ \pi_i $ będziemy oznaczali rzutowanie na {\em i-tą } współrzędną.