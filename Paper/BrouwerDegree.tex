\title{Stopień Brouwera}

\section{Lokalny stopień Brouwera}

W definicji relacji nakrywającej kluczowym pojęciem jest lokalny stopień Brouwera.
W tej pracy bedę korzystał z niego jedynie dla przestrzeni $ \mathbb R^n $ 
przytocze jednak bardziej ogólną definicje

\begin{definition} {Lokalny Stopień Brouwera.}
  Niech $X$ będzie skończenie wymiarową unormowaną przestrzenią weketorową i $ D $ będzie otwartym, ograniczonym podzbiorem $ X $.
  Niech $ f : \overline D \to X $ będzie odzorowaniem klasy $ C^1 $ i niech $c$ będzie wartością regularną tego odzorowowanie. 
  
  \begin{equation}
    f(x) \neq c  \quad \forall x \in \partial D 
  \end{equation}
to definiujemy lokalny stopień Brouwera jako : 
  
  \begin{equation}
    deg(f,D,c) := \Sigma_{x \in f^{-1}(c)\cap D} sgn(Det \mbox{ } d_x f)
  \end{equation}
  
  Jeżeli dla danego $ c $ zachodzi warunek nieosiągalności na brzegu ale nie jest wartością regularną to lokalny stopień Brouwera
  definiujemy następująco
  
    \begin{equation}
      deg(f,D,c) := deg(h,D,c)
    \end{equation}
    Gdzie $h$ jest funkcją klasy $ C^1 $ dostatecznie blisko $f$ , spełniającą warunek na brzegu i $ c $ jest jej wartością regularną.
    
    Dla $ f $ jedynie spełniającego warunek nieosiągalności na brzegu definiuejemy lokalny stopień Brouwera jako stopień 
    dowolnej funkcji klasy $ C^1 $, która spełnia warunek na brzegu i jest dostatecznie blisko f.
  

\end{definition}

Poprawność tej definicji nie jest oczywista, dowód można znaleźć w \cite{N} , \cite{Sch}.

Lokalny stopień ma następujące własności.

\begin{enumerate}
 \item Własność rozwiązania \newline
   \begin{equation}
   \mbox{Jeżeli  }  deg(f,D,c) \neq 0  \mbox{ Wtedy istnieje } x \in D  \mbox{takie, że }  f(x) = c 
   \end{equation}
   
  \item Własność homotopi.  \newline
  Niech $ H : [0,1] \times D \to \mathbb R^n $ będzie odwzorowaniem ciągłym i 
  %\begin{equation} \label{eq:homotopy}
  %
  %    \bigcup_{\lambda \in [0,1] } H^{-1}_{ \lambda }(c) \cap  \mbox{ jest zbiorem zwartym}
  % \end{equation}
  \begin{equation} \label{eq:homotopy}
    \bigcup_{\lambda \in [0,1] } H^{-1}_{\lambda }(c) \cap D \mbox{ jest zbiorem zwartym }
  \end{equation}

  Wtedy 
  \begin{equation}
      \forall \lambda \in [0,1] \quad deg(H_{\lambda},D,c) = deg(H_0,D,c) 
  \end{equation}
  
  W szczególności jeżeli $ [0,1] \times \overline{D} \subset dom(H) $ i $ \overline{D} $ jest zwarty, wtedy warunk wynika z następującego warunku
  \begin{equation}
    c \notin H([0,1], \partial{D})
  \end{equation}
  
  \item Lokalny stopień jesst funckcją lokalnie stałą \newline
    Załóżmy, że $ D $ jest otwarty i ograczniczony. Niech $p $ i $ q $ należą do tej samej spójnej składowej $ \mathbb R^n  \setminus f(\partial{D}) $  wtedy
    \begin{equation} \label{eq:locallity}
      deg(f,D,p) = deg(f,D,q)
    \end{equation}
  
  \item Własność wycinania. \newline
    Załóżmy, żę $ E \subset D $, $ E $ otwarty i 
    \begin{equation}
	f{-1}(c) \cap D \subset E 
    \end{equation}
    Wtedy
    \begin{equation}
        deg(f,E,c) = deg(f,D,c)
    \end{equation}
    
    \item Loklany stopień dla izomorfizmów afinicznych \newline
    Załóżmy, że $ f(x) = A(x-x_0) + c $, gdzie $ A $ jest izomorfizmem liniowym i $ x_0,c \in \mathbb R^n $. 
    Jeżeli $ x_0 \in D$ wtedy
    \begin{equation}
        deg(f,D,c) = sgn(Det A)
    \end{equation}
    
    \item Własność produktu. \newline
     Niech $ U_i \subset \mathbb R^{n_i}, f_i : U_i  \to \mathbb R^{n_i} $ dla $ i =1,2 $
     Wtedy odwzorowanie $ (f_1,f_2) : U_1 \times U_2 \to R^{n_1} \times R^{n_2} $ dane wzorem :
     $ (f_1,f_2)(x_1,x_2) = (f_1(x_1),f_2(x_2)) $. Wtedy dla $c_1 \in R^{n_1}  $ i $ c_2 \in R^{n_2} $ takich, że $ deg(f_1,U_1,c_1),deg(f_2,U_2,c_2) $ są zdefiniowane 
     zachodzi następująca równość 
     \begin{equation}
	deg((f_1,f_2), U_1 \times U_2 , (c_1,c_2)) = deg(f_1,U_1,c_1) \cdot deg(f_2,U_2,c_2)
     \end{equation}
     
     \item Własność mnożenia. \newline
     Niech $ D \subset \mathbb R^n $ będzie otwarty i ograniczony. Niech $ f : D \to R^n , g : \mathbb R^n \to \mathbb R^n $ są ciągłe i niech $ \Delta_i $ ozanczają 
     ograniczone spójne składowe $ \mathbb R^n \setminus f(\partial{D}$. Wtedy 
     \begin{equation}
	  deg (g \circ f ,D,p) = \Sigma_{\Delta_i} deg(g,\Delta_i,p)deg(f,D,\Delta_i) 
     \end{equation}
     
     Gdzie $ deg(f,D,\Delta_i) := deg(f,D,q_i)$ dla jakiegoś $ q_i in \Delta_i $. Z własności \ref{eq:locallity} wynika, że defnicja $ def(f,D,\Delta_i ) $ nie zależy od wyboru
     $ q_i \in \Delta_i $
     
     \item Własnośc dodwania. \newline
     Niech $ D = \bigcup_{i \in I} D_i $ , gdzie $ I = \{1,2 \dots n \} $ dla jakiegoś $ n \in \mathbb N $ i $ D_i $ są zbiorami otwartymi parami rozłącznymi takimi, żę 
     $ \partial{D_i} \subset \partial{D} $ dla każdego $ i \in I $. Wtedy dla każdego $ c \notin f(\partial{D})  $ zachodzi:
     \begin{equation}
	deg(f,D,c) = \Sigma_{i \in I} deg(f,D_i,c)
     \end{equation}
     








\end{enumerate}

